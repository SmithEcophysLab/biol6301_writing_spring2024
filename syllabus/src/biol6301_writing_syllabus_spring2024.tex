\documentclass[12pt, notitlepage]{article}   	% use "amsart" instead of "article" for AMSLaTeX format
\usepackage{geometry}                		% See geometry.pdf to learn the layout options. There are lots.
\geometry{a4paper}                   		% ... or a4paper or a5paper or ... 
%\geometry{landscape}                		% Activate for rotated page geometry
\usepackage[parfill]{parskip}    		% Activate to begin paragraphs with an empty line rather than an indent
\usepackage{graphicx}				% Use pdf, png, jpg, or eps§ with pdflatex; use eps in DVI mode
								% TeX will automatically convert eps --> pdf in pdflatex

\usepackage{hyperref}
		
%SetFonts

\usepackage[T1]{fontenc}
\usepackage[utf8]{inputenc}

\usepackage{tgbonum}

%SetFonts

\title{
	\textbf{
		BIOL 6301-029
	} \\
	\large Writing science (plant ecophysiology): How to write papers that get cited and
	proposals that get funded \\
	\large Spring 2024
}

\date{\vspace{-5ex}}

\begin{document}

{\fontfamily{phv}\selectfont %select helvetica (code = phv)

\maketitle

\section{Course Description}
This course focuses on the approaches to writing papers and proposals in the field
of plant ecophysiology. The course will be guided by the book "Writing science: 
How to write papers that get cited and proposals that get funded" by Josh Schimel (see
below). The course will involve reading discussions of book chapters as well as writing assignments.
The writing done should be for papers or proposals that the students are actively working on and,
as such, the course is designed for advanced plant ecophysiology graduate students.

\section{Expected Learning Outcomes and Objectives}
Upon completion of this class, students are expected to be able to:\par
(1)	Understand and discuss best practices for writing papers and proposals in plant ecophysiology\par
(2) Demonstrate effective writing of scientific ideas and results\par

\subsection{Class Time and Location}
Day and time: Mondays 4:00-5:20 PM, Fridays 1:00-2:20 PM

Experimental Sciences Building II (ESB II) first floor conference room.

\subsection{Instructor}
Dr. Nick Smith \par
ESB II Room 402D \par
806-834-7363 \par
nick.smith@ttu.edu \par
\textit{Meetings by appointment}

\subsection{Recommended Texts}
"Writing science: How to write papers that get cited and proposals that get funded" by Schimel (2012) \par

\section{Mode of Instruction}
All instruction will be done face-to-face.

\section{Course Materials}
All course materials, including lecture slides, readings, activities, and code 
will be posted to a GitHub repository for the course.
The primary repository address is
\url{https://github.com/SmithEcophysLab/biol6301_writing_spring2024}.

\section{Attendance Policy}
Attendance is strongly recommended. 
Course assessments will be done during class (see below).

\section{Course Assessment}
\subsection{\textit{Participation and Engagement}}
Being an active and engaged participant in the class will benefit your understanding
of material as well as your peers'. Examples include asking questions, providing feedback,
and facilitating discussion. Participation and engagement of each student will be monitored
during each class period.

\subsection{\textit{Writing assignments}}
On most weeks, students will complete and turn in a writing assignments. Assignments will
be done to build towards a completed manuscript or proposal.

\subsection{\textit{Peer evaluation}}
Students will evaluate the writing of their peers each week.

\section{Grading}
Participation and Engagement: 50\% \par
Writing assignments: 25\% \par
Peer evaluation: 25\% \par

\section{Grading Scale}
A: $\geq$ 90\% \par
B: 80 – 90\% \par
C: 70 – 80\% \par
D: 60 – 70\% \par
F: $\leq$ 59.9\% \par

\section{Missing In-class Activities}
Students will be required to be in class to receive in-class activity points. 
Please contact Dr. Smith if you plan to miss class for a university function 
\textit{prior to class}. If class is missed due to an illness, 
please let Dr. Smith know as soon as possible.

\subsection{Illness Based Absence Policy}
If at any time during this semester you feel ill, in the interest of your own health and 
safety as well as the health and safety of your instructors and classmates, you are 
encouraged not to attend face-to-face class meetings or events.  Please review the steps 
outlined below that you should follow to ensure your absence for illness will be excused. 
These steps also apply to not participating in synchronous online class meetings if you feel 
too ill to do so and missing specified assignment due dates in asynchronous online classes 
because of illness. If you are ill and think the symptoms might be COVID-19-related:
\begin{itemize}
	\item{Call Student Health Services at 806.743.2848 or your health care provider.  
	After hours and on weekends contact TTU COVID-19 Helpline at [TBA].}
	\item{Self-report as soon as possible using the Dean of Students COVID-19 webpage.
	This website has specific directions about how to upload documentation from a medical 
	provider and what will happen if your illness renders you unable to participate in 
	classes for more than one week.}
	\item{If your illness is determined to be COVID-19-related, all remaining 
	documentation and communication will be handled through the Office of the 
	Dean of Students, including notification of your instructors of the period of 
	time you may be absent from and may return to classes.}
	\item{If your illness is determined not to be COVID-19-related, please follow steps below.}
\end{itemize}

\section{Special Considerations}
Texas Tech Policies Concerning Academic Honesty, Special Accommodations for Students with 
Disabilities, Student Absences for Observance of Religious Holy Days, Accommodations for 
Pregnant Students, and other policies may be found on at this link: 
\url{https://www.depts.ttu.edu/tlpdc/RequiredSyllabusStatements.php}.

\section{Creating Livable Futures}
This class is part of a campus-wide initiative called Creating Livable Futures, 
which is sponsored in part by the Texas Tech Center for Global Communication. 
As such, one of our objectives is to prepare you to communicate, 
in a fully interdisciplinary and global way, the challenges posed by pressing issues 
that speak to our collective wellbeing and sustainability. You will be asked to translate 
and communicate the work of leading thinkers on sustainability, and to expand discussing 
those materials through research experience and experiential learning.
These objectives will be met through discussion leads and the review paper. 

Your progress in communicating about global issues will be evaluated according to the 
Center for Global Communication rubric, so you will be invited to participate 
in one or more Creating Livable Futures activities outside of class that will 
complement class content. 
Planned Creating Livable Futures activities include participating in and attending 
speaker events and conferences, edit-a-thons, blogging and publication opportunities, 
student organizations, a book club, and even small scholarship opportunities for research. 

You’ll be informed of relevant opportunities and activities as they arise over 
the course of the semester.

} %end font selection

\end{document} 
